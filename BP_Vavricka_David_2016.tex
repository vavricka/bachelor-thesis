% options:
% hidelinks remove colour boxes around hyperlinks
\documentclass[thesis=B,czech]{FITthesis}[2012/06/26]

\usepackage[utf8]{inputenc} % LaTeX source encoded as UTF-8
\usepackage{graphicx} %graphics files inclusion
% \usepackage{amsmath} %advanced maths
% \usepackage{amssymb} %additional math symbols
\usepackage{dirtree} %directory tree visualisation

\usepackage{listings}


% % list of acronyms
% \usepackage[acronym,nonumberlist,toc,numberedsection=autolabel]{glossaries}
% \iflanguage{czech}{\renewcommand*{\acronymname}{Seznam pou{\v z}it{\' y}ch zkratek}}{}
% \makeglossaries

\newcommand{\tg}{\mathop{\mathrm{tg}}} %cesky tangens
\newcommand{\cotg}{\mathop{\mathrm{cotg}}} %cesky cotangens

\department{Katedra softwarového inženýrství}
\title{Flexibilní logování pro embedded Linuxové systémy}
\authorGN{David} %(křestní) jméno (jména) autora
\authorFN{Vavřička} %příjmení autora
\authorWithDegrees{David Vavřička} %jméno autora včetně současných akademických titulů
\supervisor{Ing. Matěj Laitl}
\acknowledgements{Doplňte, máte-li komu a za co děkovat. V~opačném případě úplně odstraňte tento příkaz.}
\abstractCS{Doplňte}
\abstractEN{Sem doplňte ekvivalent abstraktu Vaší práce v~angličtině.}
\placeForDeclarationOfAuthenticity{V~Praze}
\declarationOfAuthenticityOption{4} %volba Prohlášení (číslo 1-6)
\keywordsCS{logování, vestavěné systémy, logovací démoni, Linux, Rsyslog}
\keywordsEN{logging, embedded systems, logging daemons, Linux, Rsyslog}

\begin{document}

% \newacronym{CVUT}{{\v C}VUT}{{\v C}esk{\' e} vysok{\' e} u{\v c}en{\' i} technick{\' e} v Praze}
% \newacronym{FIT}{FIT}{Fakulta informa{\v c}n{\' i}ch technologi{\' i}}

\begin{introduction}

\end{introduction}

\chapter{Technické požadavky}
Cílem je upravit logovací řešení pro set-top box EKT DID7006mTF~\cite{ekt7006} tak, aby splňovalo technické požadavky popsané v této kapitole. Řešení musí fungovat a být otestováno na zmíněném modelu set-top boxu a pokud možno by mělo být přenositelné i na jiné typy set-top boxů.
Požadavky jsou rozděleny na základní a rozšířené. Rozšířené požadavky není nutno implementovat.


\section{Základní technické požadavky}

\subsection{Snížení objemu logů}
Je žádoucí umožnit snížit objem zasílaných logů z důvodu přílišného zatížení sítě a serverových disků. Původní řešení všechny logy odesílalo na vzdálené servery. Nové řešení by mělo odesílat pouze důležité zprávy, tedy zprávy s nízkou severitou.

\subsection{Vzdálená konfigurace}
Technické řešení musí být schopno za běhu pomocí SHELL-ového API měnit minimální povolenou severitu zpráv pro jednotlivé komponenty a dále toto API musí mít výchozí severitu, která se použije pro komponenty ji nemají explicitně nastavenou. Takto změněné nastavení musí být perzistentní i po restartu STB. Výchozí nastavení se obnoví až po factory resetu. API navrhne dle své libovůle sám řešitel.

\subsection{Rate-limiting odesílaných zpráv}
Nové logovací řešení musí být schopné provádět rate-limiting odesílanych zpráv tak, aby nepřekročilo maximální vyhrazenou šířku pásma. Bude umožněno nastavit jak dlouhodobé tak krátkodobé limity. Naivní rate-limiting je i v existujícím řešení, řešitel navrhne výchozí nastavení nového řešení tak, aby přibližně odpovídalo současnému chování. 

\subsection{Formát logů}
Je nutno zachovat formát logů jako ho má původní řešení, aby se jednalo o drop-in replacement bez nutnosti jakkoli měnit konfiguraci serveru, který sbírá logy od set-top boxů.

\subsection{Razítkování zpráv}
Každé zprávě se musí přidat textový prefix if=N, kde N monotonicky roste s každou zprávou.  To slouží pro detekci ztracených zpráv. Id přeteče po 32 nebo 64 bitech, to záleží na rozhodnutí řešitele. Po rebootu STB id znovu začíná od 1.

\subsection{Post-processing zpráv}
Zadavatel má pouze částečnou kontrolu nad zprávami generovanými aplikacemi na set-top boxu, například nedokáže ve všech případech eliminovat dlouhé prefixy u zpráv. Je proto nutno takové prefixy rozpoznat a vhodně odfiltrovat před odesláním. Ze stejného důvodu mají některé zprávy nevhodně vyplněnou severitu a položku app-name. Jejich správné hodnoty jsou uloženy v textu zprávy, jejíž formát je konstantní. Řešení bude schopné tyto údaje z těla zprávy extrahovat a nahradit jimi původní metadata. Tato pravidlo musí být možné definovat a měnit bez nutnosti nového sestavení softwaru. 
Řešitel vytvoří pro ukázku 2 pravidla, která budou sloužit zadavateli jako šablony pro možná budoucí filtrovací pravidla.

\subsubsection{Pravidlo pro filtrování zpráv pro dané komponenty}
Zprávy s nastavenou severitou INFO a komponentou sld\_br je třeba změnit podle následujícího vzoru.

\subsubsection*{Originální zpráva:}
\begin{scriptsize}
\begin{verbatim}
2016-02-18T14:05:24+01:00 cc-b8-f1-00-6f-07 sld_br: id=559 
:[stbhal.cpp:debug:520]: INFO: [94mDEBUG: InformationService:
 Reading 'nangu.video.forcedScart': false[0m
\end{verbatim}}
\end{scriptsize}

\subsubsection*{Upravená zpráva:}
\begin{scriptsize}
\begin{verbatim}
2016-02-18T14:05:24+01:00 cc-b8-f1-00-6f-07 nangu-portal: [94m InformationService:
 Reading 'nangu.video.forcedScart': false[0m
\end{verbatim}}
\end{scriptsize}

\subsubsection{TODO další pravidla}
doplnit v průběhu implementace pravidel

\subsubsection{Převod severit}
Aplikace na STB používají pro logování TODO-DOPLNIT formát logů. Problémem je, že tento formát není kompatibilní s dnes za standard považovaným syslog formátem. Zadavatel proto požaduje změnit severity zpráv podle následující tabulky.

\begin{table}[ht]
\centering
	\caption{Převodní tabulka}	\begin{tabular}{|l|l|}\hline
		Portal		& Syslog			\tabularnewline \hline \hline
		ERROR: 1	& ERR			\tabularnewline \hline
		WARN: 1		& WARN			\tabularnewline \hline
		INFO: 1		& NOTICE			\tabularnewline \hline
		DEBUG: 1	& INFO			\tabularnewline \hline
		TRACE: 1	& DEBUG			\tabularnewline \hline
	\end{tabular}
\end{table}

\section{Rozšířené technické požadavky}

\subsection{Komprese zpráv}
Bylo by vhodné zvážit pro a proti komprese zpráv. Vyplatí se ušetřená přenesená data oproti režiji spojené s kompresí a dekompresí zpráv?

\subsection{Rozšíření C++ komponenty}
Zadavatel na STB provozuje malého démona dmd napsaného v C++, který mimo jiné obsahuje minimalistický HTTP webserver. Dále v browseru běží Javascript aplikace (nangu.TV portál), která pomocí messagingu komunikuje s centrálním serverem. Tato Javascript aplikace ovšem nemůže přímo používat Shell API.
Požadavkem je rozšířit C++ komponentu dmd tak, aby umožnila Javacript aplikaci řídit konfiguraci logování (viz bod Vzdálená konfigurace).

\chapter{Analýza}

\section{Set-top box O2 TV - EKT DID7006mTF}
\subsection*{Hardware specifikace}
\subsection*{Nainstalovaný software}
\subsection*{Gu}
\subsection*{PKGBUILD}

\section{Současné řešení}

\section{Volba postupu řešení}
Prvně je nutno zvážit, zda problém řešit na straně serveru nebo set-top boxu. Vhodnou konfigurací logovacího démona na straně serveru, který by nepotřebné zprávy zavčas rozpoznal, zahodil a dále nezpracovával bychom splnili požadavek na snížení zátěže serverových disků. Přetížení sítě se takto vyřešit ale nedá a proto toto řešení zavrhuji.
Je tedy nutno problém řešit na straně set-top boxu kde původní řešení je postaveno na busy-box syslogd. Nabízí se možnost upravit fungování tím způsobem, aby se logy s nízkou severitou už na set-top boxu zahazovaly a pouze v případě potřeby bylo umožněné na dálku změnit konfiguraci démona tak, aby se povolilo logování pro logy s nastavenou danou komponentou a severitou. To vše přes SHELL-ové API.
Součástí zadání je ale i implementovat škrcení zpráv, aby nedocházelo k zahlcení linky. Takovou možnost prostý syslogd neposkytuje a je proto nutno zvážit napsání vlastního démona či nasazení jiného, vyspělejšího logovacího démona.

\section{Srovnání logovacích démonů}
Démon v UNIXovém světě je označení pro takový proces, který oproti běžným procesům neintereaguje přímo s uživatelem, ale běží na pozadí operačního systému a funguje samostatně. Účelem logovacího démona je sběr logů od ostatních procesů, které následně v závislosti na jeho konfiguraci dokáže filtrovat a ukládat na disk či odesílat na požadovaný vzdálený server.

V této kapitole zmíním a popíši vybrané logovací démony a v závěru kapitoly je porovnám.

\subsection*{BusyBox Syslogd}
Tato logovací utilita se skládá ze dvou démonu, jmenovitě z Klogd, který má na starost logy linuxového kernelu, druhým démonem je pak syslogd, který spravuje všechny zbylé logy.
Oba tyto démoni mají velice omezenou funkcionalitu. Dokáží logy lokálně ukládat, přeposílat je dále po síti, zahazovat duplikáty, rotovat logy v závislosti na velikosti a tím výčet jejich funkcionalit končí.

\subsection*{Syslog-ng}
Flexibilní logovací démon zaměřený na centralizované a zabezpečené logování. Má široké možnosti nastavení a poskytuje obrovské množství funkcionalit. Takže jeho vhodným nakonfigurováním se dají snadno splnit všechny vytyčené technické požadavky až na požadavek pro možnost vzdálené změny konfigurace.
Je nutno ale zmínit, že pokročilé funkce jako například šifrování zpráv, bufferování nebo message-rate kontrola jsou dostupné pouze v komerční closed-source verzi.

\subsection*{Rsyslog}
Výčet funkcionalit Rsyslogu je ještě obsáhlejší než u Syslog-ng. Technické požadavky se s jeho použitím tedy také dají splnit všechny, kromě vzdálené změny konfigurace. Oproti Syslog-ng je Rsyslog kompletně zdarma a open-source. Navíc není jen logovacím démonem, ale i analyzérem logů. Dokáže logy podle obsahu zprávy měnit, třídit a jinak s nimy nakládat.
Že je Rsyslog vyspělý a kvalitní program dokazuje fakt, že je defaultním logovacím démonem na spoustě linuxových distribucích, jmenovitě například v Ubuntu.
Jeho slabiny shledávám v nedostatečné dokumentaci a ve specifických případech v neefektivním analyzování logů mající za následek (obvzláště na embedded zařízení s pomalým ARM procesorem) rychlostní deficit. Jeho vývoj obstarává z velké většiny pouze jeden člověk, jeho původní tvůrce Rainer Gerhards. A v jednom člověku není snadné dovést tak rozsáhlý projekt k dokonalosti.

\subsection*{Porovnání výše zmíněných logovacích utilit}
Pouhým nasazením jakéhokoli známého logovacího démonu není možné splnit všechny vytyčené technické požadavky. V případě ponechání původního BusyBox syslogd démonu by pro splnění technických požadavků bylo nutno doimplementovat tolik funkcionalit, že by to výrazně přesahovalo rozsah bakalářské práce.
Výhodněji se jeví nasadit pokročilý logovací démon jako je Syslog-ng či Rsyslog. Oba totiž poskytují námi požadované funkcionality. Syslog-ng však většinu z nich poskytuje pouze v placené closed-source verzi a proto jsem se rozhodl pro Rsyslog.

\section{Vzdálená konfigurace}
Rsyslog při svém zapnutí čte konfigurační soubor rsyslog.conf, který za jeho běhu není možné měnit. Je pro to nutné napsat SHELL-ové API, které umožní na dálku přenastavit tento konfigurační soubor a restartovat rsyslog

\subsection*{Shellové API - 1. způsob}

\begin{scriptsize}
\begin{verbatim}
set_log_verbosity.sh [component] [severity]
\end{verbatim}}
\end{scriptsize}

TODO rozepsat

\subsection*{Shellové API - 2. způsob}


\begin{scriptsize}
\begin{verbatim}
/etc/logging.conf
\end{verbatim}}
\end{scriptsize}

\begin{scriptsize}
\begin{verbatim}
component1  = DEBUG
componentXY = INFO
...
DEFAULT     = INFO
\end{verbatim}}
\end{scriptsize}

TODO rozepsat



\chapter{Realizace}

\section{Nasazení Rsyslogu}
\subsection*{build}
\subsection*{konfigurace}

\section{Shell API}

\chapter{Testování}

\begin{conclusion}

\end{conclusion}

\bibliographystyle{csn690}
\bibliography{mybibliographyfile}

\appendix

\chapter{Seznam použitých zkratek}
% \printglossaries
\begin{description}
	\item[GUI] Graphical user interface
	\item[XML] Extensible markup language
\end{description}


% % % % % % % % % % % % % % % % % % % % % % % % % % % % 
% % Tuto kapitolu z výsledné práce ODSTRAŇTE.
% % % % % % % % % % % % % % % % % % % % % % % % % % % % 
% 
% \subsection{Typografie}
% 
% Při psaní dodržujte typografické konvence zvoleného jazyka. České \uv{uvozovky} zapisujte použitím příkazu \verb|\uv|, kterému v~parametru předáte text, jenž má být v~uvozovkách. Anglické otevírací uvozovky se v~\LaTeX{}u zadávají jako dva zpětné apostrofy, uzavírací uvozovky jako dva apostrofy. Často chybně uváděný symbol "{} (palce) nemá s~uvozovkami nic společného.
% 
% Dále je třeba zabránit zalomení řádky mezi některými slovy, v~češtině např. za jednopísmennými předložkami a spojkami (vyjma \uv{a}). To docílíte vložením pružné nezalomitelné mezery -- znakem \texttt{\textasciitilde}. V~tomto případě to není třeba dělat ručně, lze použít program \verb|vlna|.
% 
% Více o~typografii viz \cite{kobltypo}.
% 
% \subsection{Obrázky}
% 
% Pro umožnění vkládání obrázků je vhodné použít balíček \verb|graphicx|, samotné vložení se provede příkazem \verb|\includegraphics|. Takto je možné vkládat obrázky ve formátu PDF, PNG a JPEG jestliže používáte pdf\LaTeX{} nebo ve formátu EPS jestliže používáte \LaTeX{}. Doporučujeme preferovat vektorové obrázky před rastrovými (vyjma fotografií).
% 
% \subsubsection{Získání vhodného formátu}
% 
% Pro získání vektorových formátů PDF nebo EPS z~jiných lze použít některý z~vektorových grafických editorů. Pro převod rastrového obrázku na vektorový lze použít rasterizaci, kterou mnohé editory zvládají (např. Inkscape). Pro konverze lze použít též nástroje pro dávkové zpracování běžně dodávané s~\LaTeX{}em, např. \verb|epstopdf|.
% 
% \subsubsection{Plovoucí prostředí}
% 
% Příkazem \verb|\includegraphics| lze obrázky vkládat přímo, doporučujeme však použít plovoucí prostředí, konkrétně \verb|figure|. Například obrázek \ref{fig:float} byl vložen tímto způsobem. Vůbec přitom nevadí, když je obrázek umístěn jinde, než bylo původně zamýšleno -- je tomu tak hlavně kvůli dodržení typografických konvencí. Namísto vynucování konkrétní pozice obrázku doporučujeme používat odkazování z~textu (dvojice příkazů \verb|\label| a \verb|\ref|).
% 
% \begin{figure}\centering
% 	\includegraphics[width=0.5\textwidth, angle=30]{cvut-logo-bw}
% 	\caption[Příklad obrázku]{Ukázkový obrázek v~plovoucím prostředí}\label{fig:float}
% \end{figure}
% 
% \subsubsection{Verze obrázků}
% 
% % Gnuplot BW i barevně
% Může se hodit mít více verzí stejného obrázku, např. pro barevný či černobílý tisk a nebo pro prezentaci. S~pomocí některých nástrojů na generování grafiky je to snadné.
% 
% Máte-li například graf vytvořený v programu Gnuplot, můžete jeho černobílou variantu (viz obr. \ref{fig:gnuplot-bw}) vytvořit parametrem \verb|monochrome dashed| příkazu \verb|set term|. Barevnou variantu (viz obr. \ref{fig:gnuplot-col}) vhodnou na prezentace lze vytvořit parametrem \verb|colour solid|.
% 
% \begin{figure}\centering
% 	\includegraphics{gnuplot-bw}
% 	\caption{Černobílá varianta obrázku generovaného programem Gnuplot}\label{fig:gnuplot-bw}
% \end{figure}
% 
% \begin{figure}\centering
% 	\includegraphics{gnuplot-col}
% 	\caption{Barevná varianta obrázku generovaného programem Gnuplot}\label{fig:gnuplot-col}
% \end{figure}
% 
% 
% \subsection{Tabulky}
% 
% Tabulky lze zadávat různě, např. v~prostředí \verb|tabular|, avšak pro jejich vkládání platí to samé, co pro obrázky -- použijte plovoucí prostředí, v~tomto případě \verb|table|. Například tabulka \ref{tab:matematika} byla vložena tímto způsobem.
% 
% \begin{table}\centering
% 	\caption[Příklad tabulky]{Zadávání matematiky}\label{tab:matematika}
% 	\begin{tabular}{|l|l|c|c|}\hline
% 		Typ		& Prostředí		& \LaTeX{}ovská zkratka	& \TeX{}ovská zkratka	\tabularnewline \hline \hline
% 		Text		& \verb|math|		& \verb|\(...\)|	& \verb|$...$|		\tabularnewline \hline
% 		Displayed	& \verb|displaymath|	& \verb|\[...\]|	& \verb|$$...$$|	\tabularnewline \hline
% 	\end{tabular}
% \end{table}
% 
% % % % % % % % % % % % % % % % % % % % % % % % % % % % 

\chapter{Obsah přiloženého CD}

\begin{figure}
	\dirtree{%
		.1 readme.txt\DTcomment{stručný popis obsahu CD}.
		.1 exe\DTcomment{adresář se spustitelnou formou implementace}.
		.1 src.
		.2 impl\DTcomment{zdrojové kódy implementace}.
		.2 thesis\DTcomment{zdrojová forma práce ve formátu \LaTeX{}}.
		.1 text\DTcomment{text práce}.
		.2 thesis.pdf\DTcomment{text práce ve formátu PDF}.
		.2 BP\_Vavricka\_David\_2016.pdf\DTcomment{text práce ve formátu PDF}.
	}
\end{figure}

\end{document}